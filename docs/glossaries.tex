\newglossaryentry{Web2}{
    name={Web2},
    description={Web3 is the name commonly used for the second generation of the internet. The term was first used at a conference between O'Reilly and MediaLive International~\autocite[1]{mandiberg_4_2020}, describing the internet of the 21st century, where advancements in technology have caused a shift away from static to interactive web pages~\autocite[6]{mandiberg_4_2020}},
    first={Web2},
    text={Web2}
}

\newglossaryentry{Web3}{
    name={Web3},
    description={Web3 is the name commonly used for the third generation of the internet. The term was first suggested by Dr. Gavin Wood and represents a new vision for web applications. Meaning a shift from centralized to decentralized applications~\autocite[xxxv]{antonopoulos_mastering_2019}},
    first={Web3},
    text={Web3}
}

\newglossaryentry{DAOg}{
    name={DAO},
    description={A Decentralized Autonomous Organization is a company without hierarchical management~\autocite[xxvi]{antonopoulos_mastering_2019}.
    This means that anyone can put forward new proposals, procure a stake in the organization and use this stake to vote on new proposals through the use of smart contracts},
}

\newglossaryentry{Blockchain}{
    name={Blockchain},
    description={A blockchain is a decentralized digital ledger that records and saves data in an immutable manner~\autocite{yaga_blockchain_2018}.
    \textcite[159]{antonopoulos_mastering_2017} describes blockchains as \enquote{an ordered back-linked list of blocks of transactions}, hence their name},
    plural={blockchains},
    text={blockchain}
}

\newglossaryentry{SmartContract}{
    name={Smart Contract},
    description={Smart contracts are essentially computer programs that are stored and executed on a blockchain network enabling organizations to execute self-enforcing contract clauses~\autocite{udokwu_state_2018}.
    Deployment and execution of smart contracts are registered as immutable transactions on the relevant network},
    plural={smart contracts},
    text={smart contract}
}

\newglossaryentry{P2Pg}{
    name={P2P},
    description={A peer-to-peer network is a network of computers that exchange information among themselves making them \enquote{peers}~\autocite[3]{schoder_peer--peer_nodate}.
    Communication between nodes happens autonomously and directly without the need for centralized coordinators.
    According to \textcite[4]{schoder_peer--peer_nodate} the early internet was a peer-to-peer network},
}

\newglossaryentry{POWg}{
    name={PoW},
    description={Proof of work is a consensus mechanism that requires all nodes in a distributed network such as Bitcoin to solve a mathematical problem to be permitted to add a new data block to the blockchain~\autocite{udokwu_state_2018}.
    In the case of Bitcoin, the proof of work involves scanning for a value whose hashed output begins with a number of zero bits~\autocites[188-195]{antonopoulos_mastering_2017}[3]{nakamoto_bitcoin_2008}},
}

\newglossaryentry{POSg}{
    name={PoS},
    description={Proof of stake is a consensus mechanism that requires the staking of the network's native currency to be allowed to validate and add a new block to the blockchain~\autocites[4]{jafar_blockchain_2021}[chapter 1]{udokwu_state_2018}},
}

\newglossaryentry{POHg}{
    name={PoH},
    description={Proof of history is a consensus mechanism that uses a cryptographic verifiable delay function to hash transaction events, thereby creating a historical record where the order of events is cryptographically verifiable~\autocite{yakovenko_proof_2020}},
}

\newglossaryentry{PBFTg}{
    name={PBFT},
    description={Practical Byzantine fault tolerance is a consensus mechanism used in permissioned blockchain networks.
    In permissioned blockchain networks nodes are whitelisted by the controlling entity which means compromises on byzantine fault tolerance are possible in order to achieve higher transaction throughput.
    Subsequently, PBFT assumes that less than one-third of nodes in the network are compromised as they need to be whitelisted to gain access to the network~\autocite[1]{sukhwani_performance_2017}},
}

\newglossaryentry{GenesisBlock}{
    name={Genesis Block},
    description={The genesis block initializes the blockchain as the first block in the chain. Every subsequent block is added behind the genesis block~\autocites[162]{antonopoulos_mastering_2017}[xxix]{antonopoulos_mastering_2019}},
    first={genesis block},
    text={genesis block}
}

\newglossaryentry{MerkleTree}{
    name={Merkle Tree},
    description={\textcite[53]{yaga_blockchain_2018} describe a merkle tree as \enquote{a data structure where the data is hashed and combined until there is a singular root hash that represents the entire structure,} the merkle tree root. The concept was first proposed and patented by \textcite{merkle_method_1982} and looks like a tree when displayed in a diagram},
    first={merkle tree},
    text={merkle tree}
}

\newglossaryentry{PKg}{
    name={PK},
    description={The private key gives its owner control over the corresponding wallet address~\autocite[63]{antonopoulos_mastering_2017}},
}

\newglossaryentry{PBKg}{
    name={PBK},
    description={The public key is used to receive funds as its digital fingerprint represents a wallet address~\autocite[61]{antonopoulos_mastering_2017}},
}

\newglossaryentry{Nonce}{
    name={Nonce},
    text={nonce},
    description={A nonce is a variable number applied in a proof of work hashing algorithm.
    Nodes search for a hashed output that is smaller than the difficulty target by iterating over the nonce thereby varying the algorithms hashed output~\autocite[190]{antonopoulos_mastering_2017}},
}

\newglossaryentry{IOg}{
    name={I/O},
    description={Input/output refers to the communication between processing systems.
    The input is the data received by one system, while the output is the data it sends after processing the input},
}

