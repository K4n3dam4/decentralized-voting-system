\chapter{Introduction}\label{ch:intro}

Elections are the cornerstone modern democracies are built on and thus are arguably one of the most critical aspects of our democratic processes.
Consequently, ensuring fair elections and voting results that are trustworthy should be of concern to every democratic nation and its citizens.
However, no system is perfect, and analogous voting systems currently employed by democratic countries are no exception to this rule.

The consensus view is that voting systems must be secure, anonymous, transparent, and scalable
\autocites{lowry_desirable_2009}[5]{agora_agora_nodate}[9-11]{jafar_blockchain_2021},
which are properties that may not be impossible but very costly to achieve;\ even for industrialized nations.
Furthermore, even when a government allocates the necessary funds to secure its election processes, there is still no
guarantee that this will ensure voters’ trust in the election result, as evidenced by presidential elections in the \Gls{USA}, where voter fraud theories dominated online discussions and news reports in the weeks and months following the 2016 and 2020 elections.
In the case of the \Gls{USA}, it could be argued that the distrust was partly caused by the fact that even though regular citizens can personally audit paper ballots, they usually do not possess the necessary skills to make statistical assessments.
At the same time, the voting process consists of several steps involving election officials or centralized software, which might be intransparent for a broad part of the population, thus providing critics with ample ground for speculation about election fraud.
Once such theories have gained traction, experience has shown that their spread is not easy to stop.
Therefore, many Democrats in the \Gls{USA} still believe that the 2016 election was manipulated~\autocite{sinclair_its_2018}
even though the Mueller investigation has found no evidence of Russian interference~\autocite{mueller_report_2019}.
Similarly, many Republicans still believe widespread voter fraud is the reason, Donald.\ J. Trump lost the 2020 election.
However, statistical analysis of voter ballots does not confirm that hypothesis~\autocite{eggers_no_2021}.

A \Gls{Web3} application using blockchain technology could solve those problems while at the same time being considerably less vulnerable to cyberattacks than currently employed centralized e-voting systems.
According to~\textcite{yaga_blockchain_2018}, blockchains are decentralized digital ledgers that enable a community of users to record transactions publicly.
This makes blockchain technology an ideal candidate for electronic voting systems as transaction records are immutable and auditable.
Furthermore, the decentralized manner in which blockchain ledgers are distributed means no single point of attack exists.
Provided governments will either run a potential voting systems on their own blockchains with a high enough number of nodes or an existing and widely distributed blockchain network, e.g., Ethereum.

\section{Objectives}\label{sec:objectives}

This thesis will discuss the operational features of blockchain technology and describe the development process of a decentralized electronic voting system that has this technology at its core.
The resulting prototype will then be evaluated, and its feasibility discussed following implementation in chapter~\ref{ch:implementation}.

\subsection{Qualitative objectives}\label{subsec:qualitative-objectives}

Among others, \textcites{lowry_desirable_2009}[5]{agora_agora_nodate}[9-11]{jafar_blockchain_2021} name security, anonymity, transparency, and scalability as the most important qualitative features of a voting system.
Consequently, they will be critical qualitative considerations during implementation and evaluation.

\subsection{Quantitative objectives}\label{subsec:quantitative-objectives}

\subsection{Out-of-scope objectives}\label{subsec:out-of-scope-objectives}

%\begin{table}[h!]
%    \center
%    \begin{tabular}{ |p{3.5cm}||p{3cm}|p{3cm}|p{3cm}| }
%        \hline
%        \multicolumn{4}{|c|}{Country List} \\
%        \hline
%        Country Name or Area Name & ISO ALPHA 2 Code & ISO ALPHA 3 Code & ISO numeric Code\\
%        \hline
%        Afghanistan   & AF    &AFG&   004\\
%        Aland Islands&   AX  & ALA   &248\\
%        Albania &AL & ALB&  008\\
%        Algeria    & DZ & DZA &  012\\
%        American Samoa & AS  & ASM &016\\
%        Andorra& AD  & AND   &020\\
%        Angola& AO  & AGO&024\\
%        \hline
%    \end{tabular}
%    \caption{Qualitative development goals}
%    \label{tab:qualitative-objectives}
%\end{table}

\section{Relevance}\label{sec:relevance}

Decentralized autonomy, specifically decentralized voting, is not an entirely new idea.
The concept was first introduced by members of the Ethereum community in 2016 when they created \emph{The DAO} \autocites[section 3.1]{el_faqir_overview_2020}{falkon_story_2018}, which would become the first \gls{DAO}.
\emph{The DAO} effectively enabled its users to crowdfund community projects by investing Ethereum in exchange for DAO tokens and then using those tokens to vote on the most promising proposals~\autocite{falkon_story_2018}.

In the years since, \glspl{DAO} have become an essential part of the cryptocurrency market’s ecosystem;
however, the concept of decentralized voting has thus mainly been studied by private individuals rather than governments.
Furthermore, most companies that work on decentralized voting systems made for nations are profit-oriented and therefore do not have an open-source code base.
Hence, an open-source project like this could provide new actionable insights for governments wishing to research this topic to facilitate progress.


