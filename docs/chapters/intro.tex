\chapter{Introduction}\label{ch:intro}

Elections are the cornerstone modern democracies are built on and thus are arguably one of the most critical aspects of our democratic processes.
Consequently, ensuring fair elections and voting results that are trustworthy should be of concern to every democratic nation and its citizens.
However, no system is perfect, and analogous voting systems currently employed by democratic countries are no exception to this rule.

The consensus view is that voting systems must be secure, anonymous, transparent, and scalable
\autocites{lowry_desirable_2009}[5]{agora_agora_nodate}[9-11]{jafar_blockchain_2021},
which are properties that may not be impossible but very costly to achieve;\ even for industrialized nations.
Furthermore, even when a government allocates the necessary funds to secure its election processes, there is still no
guarantee that this will ensure voters’ trust in the election result, as evidenced by presidential elections in the \Gls{USA}, where voter fraud theories dominated online discussions and news reports in the weeks and months following the 2016 and 2020 elections.
In the case of the \Gls{USA}, it could be argued that the distrust was partly caused by the fact that even though regular citizens can personally audit paper ballots, they usually do not possess the necessary skills to make statistical assessments.
At the same time, the voting process consists of several steps involving election officials or centralized software, which might be intransparent for a broad part of the population, thus providing critics with ample ground for speculation about election fraud.
Once such theories have gained traction, experience has shown that their spread is not easy to stop.
Therefore, many Democrats in the \Gls{USA} still believe that the 2016 election was manipulated~\autocite{sinclair_its_2018}
even though the Mueller investigation has found no evidence of Russian interference~\autocite{mueller_report_2019}.
Similarly, many Republicans still believe widespread voter fraud is the reason, Donald.\ J. Trump lost the 2020 election.
However, statistical analysis of voter ballots does not confirm that hypothesis~\autocite{eggers_no_2021}.

A \Gls{Web3} application using blockchain technology could solve those problems while at the same time being considerably less vulnerable to cyberattacks than centralized e-voting systems currently in use.
According to~\textcite{yaga_blockchain_2018}, blockchains are immutable and decentralized digital ledgers that enable a community of users to record transactions publicly while preventing data tampering by bad actors.
This makes blockchain technology an ideal candidate for electronic voting systems as transactions will be stored securely on the public ledger, making them unalterable and auditable.
Furthermore, the decentralized manner in which blockchain ledgers are distributed means no single point of attack exists.
Provided governments either run their respective voting systems on their blockchains with a high enough number of nodes or an existing and widely distributed blockchain network, e.g., Ethereum.

This thesis will discuss the operational features of blockchain technology and describe the development process of a decentralized electronic voting system with this technology at its core.
The resulting product will then be evaluated, and its feasibility discussed following the implementation section.

\section{Objectives}\label{sec:objectives}

\section{Relevance}\label{sec:relevance}

Decentralized voting is not entirely new, as the concept of \Glspl{DAO}


