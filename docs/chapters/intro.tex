\chapter{Introduction}
\label{ch:intro}

Elections are the corner stone modern democracies are built on and thus are arguably one of the most important aspects of our democratic processes.
Consequently, ensuring fair elections and voting results that are trustworthy should be of concern to every democratic nation and its citizens.
Unfortunately, no system is perfect and analogous voting systems that are currently employed by democratic countries are no exception to this rule.
The consensus view is that voting systems must be secure, anonymous, transparent and scalable~
\autocites{lowry_desirable_2009}[5]{agora_agora_nodate}[9-11]{jafar_blockchain_2021}, which are properties that may not be impossible but very costly to achieve;
even for industrialized nations.
Furthermore, even when a nation does allocate the necessary amount of funds to secure its electoral processes this is still no guarantee that voters' trust in the election result will be secured.
This becomes evident when looking at the past two presidential elections of the \Gls{USA}, where theories of voter fraud dominated online discussions and news-reports in the weeks and months following both the 2016 and 2020 elections.

%\autocite{poblet_linked_2017, diaz-santiso_e-voting_2021, canada_blockchain_2020} which can be challenging to achieve even in industrialized nations.



\section{Sed elementum}\label{sec:sedelemntum}

Sed elementum nec velit sit amet viverra.
Pellentesque dignissim consectetur dolor id vulputate.
Duis semper at mi ut varius.
Phasellus finibus euismod velit eget fermentum~.
Sed a egestas ante, vel \strong{\textit{lobortis felis}}.
Nulla dignissim tincidunt lacus, a lacinia lacus posuere nec.
Orci varius natoque penatibus et magnis dis parturient montes, nascetur ridiculus mus~.

\subsection{Nulla dignissim tincidunt lacus}\label{subsec:nulladinissim}

\lipsum[1-10]

\section{fermentum}\label{sec:fermentum}

\lipsum[1-5]

\Gls{API} is in \Gls{NYC}


