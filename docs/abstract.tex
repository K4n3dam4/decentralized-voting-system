\begin{abstract}
    The implementation of online voting systems has been a topic of discussion for many years, as they could potentially eliminate the requirement for physical attendance at polling stations, allowing voters to cast their ballots from any location with an internet connection.
    However, their most significant benefit, their connection to the internet, has also been met with tremendous amount of criticism due to the risk of cyberattacks.
    Consequently, public and scientific debate on this issue has primarily resulted in the conclusion that such systems are unfeasible, despite potential benefits such as increased voter turnout and significant reductions in organizational costs.
    This changed with the emergence of Bitcoin and the subsequent launch of Ethereum, which made it possible to execute immutable code using a Turing-complete state machine acting under consensus.

    This thesis provides an overview of key features of voting systems and blockchain technology in the context of a blockchain-based implementation of such systems.
    We aimed to investigate the feasibility and challenges of developing and operating such systems and successfully created an electronic voting system that processes ballots decentrally on the blockchain while incorporating most of the qualitative features we identified during our research.
    However, we encountered challenges in addressing scalability issues and balancing security and anonymity with accessibility.
    We were unable to design a system that can process thousands of voters within a short timeframe.
    Nevertheless, we concluded that the benefits of our system in other relevant areas, including low operational costs and superior auditability, justify further research into the development of decentralized electronic voting systems.
\end{abstract}