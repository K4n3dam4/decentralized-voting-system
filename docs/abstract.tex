\begin{abstract}
    Online voting systems have been discussed since the dawn of the internet as they could eliminate the requirement of physically visiting election centers and enable voters to cast their ballots online from wherever there is an Internet connection.
    However, their most significant benefit, their connection to the internet, has also been met with a tremendous amount of criticism since this would entail that hackers could launch an attack without ever gaining physical access to the system itself.
    Consequently, public and scientific debate on this issue has primarily resulted in the conclusion that such systems are an unfeasible idea, even though they could potentially increase voter turnout and significantly reduce the organizational costs of elections.

    This all changed with the invention of Bitcoin and the subsequent launch of Ethereum, which for the first time in the history of information technology, provided the possibility to write immutable code executed by a Turing-complete state machine acting under consensus.
    As a result, the debate on online e-voting systems has again been gaining traction among commentators and researchers alike.

    Therefore, this paper provides an overview of critical qualitative features of voting systems in the context of a blockchain-based implementation of electronic voting systems.
    We successfully developed an electronic voting system that processes ballots decentrally on the blockchain and possesses most of the qualitative features we found frequently mentioned during our research.
    However, solving scalability issues and weighing security and anonymity against accessibility proved to be difficult tasks.
    Unfortunately, we could not design a system that is potentially able to process thousands of voters in a short enough timeframe.
    Despite this, we concluded that our system's benefits in other relevant areas, such as very low operational costs and superior auditability, warrant further research into developing decentralized electronic voting systems.
\end{abstract}